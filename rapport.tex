\documentclass[11pt]{scrreprt}
\usepackage[T1]{fontenc}
\usepackage[utf8]{inputenc}
\usepackage[french]{babel}
\usepackage[scale=0.775]{geometry}
\usepackage{lmodern}
\usepackage[ilines]{scrpage2}
\usepackage[pdftex, bookmarks=true, hidelinks]{hyperref}
\usepackage{graphicx}
\usepackage{tocbibind}
\usepackage{chngcntr}
\usepackage{tabularx}
\usepackage{float}
\usepackage{scrhack}
\usepackage{ulem}
\usepackage{enumitem}

\counterwithout{figure}{chapter}
\counterwithout{table}{chapter}
\pagestyle{scrheadings}

% clear head and foot
\clearscrheadings
\clearscrplain
\clearscrheadfoot

\cefoot[\textsc{Nathan Raspe \& Matteo Taroli}]{\textsc{Nathan Raspe \& Matteo Taroli}}
\cofoot[\textsc{Nathan Raspe \& Matteo Taroli}]{\textsc{Nathan Raspe \& Matteo Taroli}}
\lefoot[]{}
\lofoot[]{}
\refoot[\thepage]{\thepage}
\rofoot[\thepage]{\thepage}

\begin{document}

    \renewcommand{\labelitemi}{$\bullet$}
    \renewcommand{\labelitemii}{$\circ$}
    %%%%%% TITLE PAGE %%%%%%%%%%%%%%%%%%%
    \begin{titlepage}
        \begin{center}
            \includegraphics[width=10cm]{images/logo.png}
            ~\\[1.5cm]

            % Title
            \rule{\textwidth}{1pt} \\[0.4cm]
            \Huge{\textsc{\textbf{History Pub}}}\\ \Large{\textit{Découvrez Soignies en boisson}\\[0.4cm]}

            \rule{\textwidth}{1pt} \\[1.5cm]

            \textsc{\Large Projet de Développement d'Application Mobile}\\[0.5cm]

            % Authors
            Nathan Raspe \& Matteo Taroli

            \vfill

            {\large 27 Novembre 2015}
            \vfill
        \end{center}
    \end{titlepage}

    % Table of contents
    \pagenumbering{roman}
    \tableofcontents
    % List of figures
    \renewcommand\listfigurename{Table des illustrations}
    \listoffigures
    \pagebreak
    \pagenumbering{arabic}

    %%%%%%%% BEGIN CONTENT %%%%%%%%%%%%
    \chapter{Introduction}
    Dans le cadre du cours de Dévelopement d'Application Mobile, nous avons du développer une application reprenant le concept du jeu de piste. Cette application, nommée \textsc{History Pub} vous propose de découvrir la ville de Soignies en passant par les différents bars que la ville offre.\\

    Sur le chemin entre ces différents établissements, vous aurez l'occasion d'en apprendre plus sur l'histoire et le folklore de la ville. Durant les différents arrêts, vous aurez bien entendu l'occasion de profiter des bar et des diverses boissons proposées avant de reprendre le chemin vers d'autres découvertes!\\

    Attention tout de même, l'alcool est à consommer avec modération, sous peine de ne plus pouvoir répondre aux questions données!

    \chapter{Mode d'emploi}
    L'utilisation de \textsc{History Pub} est très facile. Une fois l'application lancée, le jeu ne démarre réellement qu'une fois que vous vous trouvez dans la zone de jeu, qui vous est logiquement donnée.

    % Afficher une screenshot de EtapeActivité demandant de se déplacer dans la zone

    Une fois dans la zone, et après avoir lu une courte description de l'endroit et où de son histoire, vous vous retrouverez devant une épreuve qui peut prendre différentes formes. Ces épreuves sont décrites ci-dessous.

    \section{Caractéristiques communes aux épreuves}
    Lors des épreuves de question à choix multiples ou de question ouvertes, votre réponse doit être vérifiée. Cette vérification n'est faite qu'une fois la réponse validée par l'utilisateur, c'est à dire une fois que le bouton OK est pressé. Cette vérification peut mener à trois différents toast possibles :
    \begin{description}[style=nextline]
        \item[Pas de réponse donnée]Le toast invitera l'utilisateur à répondre à la question.
        \item[Réponse correcte] Le toast félicite l'utilisateur et affiche le nombre de points gagnés.
        \item[Réponse incorrecte] Le toast se désole de la mauvaise réponse et donne la réponse correcte, permettant à l'utilisateur d'apprendre de ses erreurs.
    \end{description}

    \section{Epreuve Question à Choix Multiple}
    % mettre un screenshot

    Pour répondre à une question à choix multiples, il vous suffit de cliquer sur la réponse que vous pensez correcte. Vous pouvez cliquer n'importe où sur la ligne de ce choix, que ce soit sur la checkbox ou sur le texte de la réponse proposée.\\

    Un symbole $\surd$ s'affiche dans la checkbox de la réponse sélectionnée. Si celle ci vous convient, vous pouvez confirmer votre choix grâce au bouton OK. Sinon, vous êtes libre de changer ce choix autant de fois que vous le désirez.\\

    \section{Epreuve Question Ouverte}
    % mettre un screenshot

    Pour répondre à une question ouverte, il vous suffit de taper la réponse demandée. Cette réponse est unique mais peut être entrée de différente manière. Par exemple pour une question dont la réponse attendu serait \og Institut Paul Lambin\fg, la réponse \og IPL\fg serait aussi acceptée.\\

    Une fois la réponse entrée, vous pouvez confirmer votre solution avec la bouton entré de votre clavier virtual ou, après avoir quitté ce clavier, avec la bouton OK.

    \section{Epreuve Photographie}
    % mettre un screenshot

    Pour répondre à une question phorotgaphique, il vous suffit de vous trouver dans la zone requise et de prendre une photo du batiment ou autre objet demandé. Une fois la photo prise, vous avez un aperçu de l'image et il ne vous reste plus qu'à accepter cette photo.\\

    Aucune vérification n'est faite sur ce type d'épreuve. Faites donc de votre mieux, on compte sur vous!

    \section{Fin du jeu}
    Une fois tout le parcours fini, vous serez face à un écran récapitulatif vous permettant de savoir le nombre de points que vous avez marqués ainsi que le temps que vous avez mis pour finir ce parcours.\\

    Cette page vous permet aussi de partager votre score avec vos connaissance ainsi que de les mettre au défi de vous battre!

    \chapter{Architecture}

    \chapter{Bugs connus}
    % Aucun bug connu pour le moment :P

    \chapter{Utilisation de code open-source}
    Le code de \textsc{History Pub} n'utilise pas réellement de code open-source à proprement parler. Cependant, certaines parties sont fortement inspirées, voire même simplement reprises, des exemples et tutoriels fournis par \textsc{Google} sur leur plateforme \url{developer.android.com}.\\

    La partie géolocalisation par exemple est principalement une copie de leur exemple, adapté à nos besoins.\\

    De plus, le CSS utilisé pour les boutons des parties HTML est fortement basé sur le codepen fait par RayCh et trouvé à cette adresse : \url{http://codepen.io/iraycd/pen/dHrxv}.

    \chapter{License}

    \noindent \textsc{History Pub} est placé sous license GNU GENERAL PUBLIC LICENSE version 3.
    \hfill\\

    \noindent Cela signifie que n'importe qui \textbf{peut} :
    \begin{itemize}
        \item utiliser notre code de façon privée
        \item utiliser notre code dans des projets à but commercial
        \item redistribuer notre code
        \item modifier notre code
    \end{itemize}
    \hfill\\
    Mais cela signifie aussi qu'une personne utilisant tout ou partie du code de \textsc{History Pub} dans un projet \textbf{doit} :
    \begin{itemize}
        \item publier les sources de ce projet
        \item inclure une copy de la license et copyright dans ce projet
        \item préciser les changements importants fait à notre code
    \end{itemize}
    \hfill\\
    Celà signifie aussi que notre code est fourni sans garantie et que nous ne pouvons donc pas être tenus responsable de problèmes quelconques liés à son utilisation en dehors de \textsc{History Pub}.
    \hfill\\

    \noindent Pour plus d'information, la license complète peut être trouvée à l'adresse suivante : \url{https://www.gnu.org/licenses/gpl.html}

\end{document}
